\chapter{Discussion}
\label{ch:discussion}

\section{Synthesis of Findings}
The literature review reveals a mature body of work on admittance control for physical human-robot interaction, with recent advances in adaptive/variable strategies that respond to human intent. However, a critical examination exposes a blind spot: the vast majority of studies assume a \textit{known} constraint geometry.

This assumption is problematic for real-world industrial tasks where operators guide tools across surfaces of arbitrary shape---think of sanding a curved car body panel or grinding corrosion off a pipe elbow. In these scenarios, the surface normal vector $\mathbf{n}$ is not available from a CAD model; it must be inferred in real-time from sensor data (e.g., force/torque measurements, velocity directions).

\section{Positioning of the Proposed Research}
The proposed thesis sits at the intersection of two established fields:

\begin{itemize}
    \item \textbf{Adaptive Admittance Control:} Well-studied for free motion and simple contacts. Provides the ``power assist'' capability.
    \item \textbf{Force/Position Hybrid Control:} Classical approach for constrained motion, but typically assumes known constraint frames.
\end{itemize}

By combining online surface estimation with direction-dependent (anisotropic) compliance, the thesis proposes a novel control architecture that:
\begin{enumerate}
    \item Estimates the local surface normal from force and velocity measurements.
    \item Projects the admittance controller into the estimated tangent plane.
    \item Maintains a desired contact force along the estimated normal (``cruise control'').
    \item Provides low apparent mass in the tangent plane (``power assist'').
\end{enumerate}

\section{Comparison with Closest Existing Work}
The closest existing work is the anisotropic variable force guidance for handovers \cite{anisotropic2024handover}. However, key differences exist:
\begin{itemize}
    \item That work addresses \textit{transient} handover motions, not \textit{sustained} surface-following tasks.
    \item The ``anisotropy'' is defined in a fixed task frame, not an online-estimated surface frame.
    \item No force-hold (``cruise control'') functionality is implemented.
\end{itemize}

\section{Limitations of This Review}
\begin{itemize}
    \item The search was limited to English-language publications in major databases.
    \item Grey literature (patents, technical reports) was not systematically included.
    \item The Active Inference section is less comprehensive due to the nascent state of the field.
\end{itemize}

\section{Implications for the Thesis}
The identified gap (anisotropic control on unknown surfaces) is both \textbf{technically challenging} and \textbf{practically relevant}. The key technical sub-problems to address are:
\begin{enumerate}
    \item \textbf{Surface Normal Estimation:} How to robustly estimate $\mathbf{n}$ from noisy force/velocity data.
    \item \textbf{Frame Projection:} How to project the admittance dynamics into the tangent plane without introducing discontinuities.
    \item \textbf{Stability:} How to ensure passivity/stability when the reference frame is time-varying.
\end{enumerate}

These sub-problems will form the core technical contributions of the thesis.

